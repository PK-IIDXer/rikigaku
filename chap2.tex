\chapter{運動の法則}

質点に対して、いくつかの仮定(公理)を追加します。
力学はここから始まるのです。
まずは運動の状況を見ていきましょう。



\section{簡単な運動}

最も簡単な運動は「静止」している状態でしょう。
\begin{definition}
  質点の運動$\gamma:I\to\mathbb{R}^3$が定値写像になっているとき、質点$\gamma$は\textbf{静止}しているという。
  静止している質点の運動は、静止であると言い表す。
\end{definition}

静止の次に簡単な運動は、速さが変化しない運動、つまり「等速直線運動」でしょう。
\begin{definition}
  質点の運動$\gamma:I\to\mathbb{R}^3$に対し、$\dot\gamma:I\to\mathbb{R}^3$が定値写像になっているとき、質点$\gamma$は\textbf{等速直線運動}しているという。
  等速直線運動している質点の運動は、等速直線運動であると言い表す。
\end{definition}

質点が静止しているならば、等速直線運動しているということが言えます。
\begin{theorem}
  質点$\gamma:I\to\mathbb{R}^3$の運動が静止であるならば、その運動は等速直線運動である。
\end{theorem}
\begin{proof}
  $\gamma$は静止しているので、任意の2時刻$t_0,t\in I$において$\gamma(t)=\gamma(t_0)$。
  ゆえに
  \[
    \dot\gamma(t_0)=\lim_{t\to t_0}\frac{\gamma(t)-\gamma(t_0)}{t-t_0}=\lim_{t\to t_0}0=0
  \]
  よって$\gamma$は等速直線運動である。
\end{proof}

また、等速直線運動する質点の加速度$\ddot\gamma$は0への定値写像であることが同値であることも言えます。
ここには\textbf{平均値の定理}を用いるのですが、それはのち(§\ref{sec:mean-value-theorem})に証明しましょう。
いったん平均値の定理は認めて、以下を証明しましょう。
\begin{theorem}
  質点$\gamma$の運動が等速直線運動であることと、任意の$t_0\in I$に対して$\ddot\gamma(t_0)=0$は同値である。
\end{theorem}
\begin{proof}
  質点$\gamma$の運動が等速直線運動であるとすると、任意の$t,t_0\in I$に対して、$\dot\gamma(t)=\dot\gamma(t_0)$。
  ゆえに
  \[
    \ddot\gamma(t_0)=\lim_{t\to t_0}\frac{\dot\gamma(t)-\dot\gamma(t_0)}{t-t_0}=0
  \]
  ゆえに$\ddot\gamma(t_0)=0$ (${}^\forall t_0\in I$)。

  逆に$\ddot\gamma(t_0)=0$ (${}^\forall t_0\in I$)とする。
  $\gamma$を成分ごとに分けて、$\gamma(t):=(\gamma_x(t),\gamma_y(t),\gamma_z(t))$で考える。
  このとき任意の$t_1,t_2\in I$、$t_1<t_2$に対して、平均値の定理から、ある$c\in(t_1,t_2)$が存在して、
  \[
    \frac{\dot\gamma_i(t_1)-\dot\gamma_i(t_2)}{t_1-t_2}=\ddot\gamma_i(c)=0,\quad(i=x,y,z)
  \]
  をみたす。すなわち$\dot\gamma_i(t_1)=\dot\gamma_i(t_2)$なので、$\dot\gamma_i(t)$は定数写像となる。
\end{proof}



\section{力の定義}

「力」という概念の定義は非常に難しいものです。
アリストテレスは「槍を投げると、飛んでいる間は運動方向に力が働き続ける。その力が徐々に失われて行って、しまいには地面に落ちる」というような考え方をしていたようです。

実際のところ「力」って何なのさ?という問題は哲学の分野で語ってもらうことにして、物理学ではとにかく\textbf{定量化して計算できること}を重視します。
言い換えれば、そのように現実世界を「モデリング」して、定量的な計算や未来予測を可能にしようぜ、というのがニュートンの思想の一端ではないでしょうか。

\subsection{身近な例}

椅子を引いたり、押し戻したりするとき、僕たちは「椅子に力をかけた」と思うでしょう。
これはなぜそう思うのでしょうか?
いくつか考えられる理由がありますが、大きく分けて以下の二種類あると思います。
\begin{itemize}
  \item それなりに筋肉を使って、疲れるから
  \item 椅子が動いたから
\end{itemize}
どちらも一理あります。

では、自由落下するボールはどうでしょうか?
現代人は「地球からの重力が働いている」と理解していますが、ニュートン以前の科学者の気持ちになってみると、上記の「椅子の例」とは若干異なっているように思うかもしれません。
というのも、力というのは、人間ないし動物などが体を触れてものを動かすときに働くように思えるからです。

しかしこの自由落下するボールにも、椅子の例と同様の考え方で、力が働いていると考える方法もあるにはあります。
これらはどちらも
\begin{center}
  加速度が時間変化している
\end{center}
のです。

椅子の例だと、静止している状態(つまり速度0の状態)から力をかけることで、速度に変化を与え、結果としてその位置を移動させています。
自由落下するボールはもっと単純で、観測事実としておよそ時間の2乗に比例して速度が速くなっていくことが分かっていました。

結局のところ、自由落下するボールには力が働いているのか、いないのか?
\begin{center}
  いったん働いているとして考えてみて、うまくいったら儲けもん!
\end{center}
こういう考え方をしてみましょう。
もし自由落下するボールにも力が働いているのであれば、力とは質点の運動に対して、\textbf{その加速度を変化させる原因になるもの}であるはずです。

仮にそうであるとして、どうすれば「うまくいった」と言えるでしょうか?
物理学は常に、先行研究や実験事実をフォローすることを求められます。
先行研究や実験事実をフォローすれば、うまくいったと言えるでしょう。

\subsection{天体の運行}

ニュートンの時代には、天体の観測技術が十分高まっており、星々の運行に関する膨大な研究データが存在していたようです。
それらのデータから、ケプラーは有名な3法則を導きました。
ニュートンは力学を考えるに当たって、おそらく、ケプラーの法則のことを考えていたんだと思います。

先ほどの身近な例で言った、「自由落下するボールに力が働いている」と仮定するとしたら、加速度の変化は力の存在を意味します。
ケプラーの第一法則によれば、惑星は太陽の周りを楕円運動しており、これは加速度が変化する運動なので、惑星には、どこからかの力が働いているはずです。
またこの楕円運動の焦点の一つが太陽なので、惑星にかかっている力の有力候補は、太陽ぐらいしかないです。
あとはこの力の向きや大きさを見積もって、ケプラーの法則に近づけられれば!?
あるいはさらに、同じ法則で自由落下するボールの運動も記述できれば、「うまくいった」と言えるんじゃないでしょうか?

…「ニュートンのりんご」って、こういう感じの話なんじゃないですかね、知らんけど。

\subsection{力の定義}

もし太陽と惑星の関係と同じように、自由落下するボールの運動も記述できるとするならば、ボールに対する「太陽」はなんでしょう?
それはもう「地球」しかないでしょう。
いずれにしても、力というのは質点単体で考えることはできず、二つ以上の質点の相互作用のことを力と呼んでいるのかもしれません。
\textbf{いや、そういうことにしましょう}。
そういえば椅子を引く例にしても「手」と「椅子」の間の関係で成り立つ現象ですし、振り子をブラブラさせるのだって「手」と「紐」と「振子」と「重力」の間の現象です。

以上を踏まえて、力は次のように、ふわっと定義します。
\begin{definition}
  力とは、質点間の相互作用を定量的に表すベクトル量である。
\end{definition}
「力」を「相互作用」に言い換えただけのように見えますが、「定量的なベクトル量」ということにしたことに意味があります。
これは「\textbf{力の線形和}」を考える妥当性があるよと宣言したことにもなります。
そのため、質点に複数の力がかかっているとき、その力の総和を「\textbf{合力}」と言い表すこともあります。

また、「加速度を変化させるものを力と呼びたい」という考えから、次のような力も考えなければならないです。
自由落下するボールは加速度を変化させるので、何かの力が働いているとみなすことになりましょう。
これを仮に\textbf{重力}と呼びましょう。
しかし机に置いてあるボールは、加速度を変化させません。
これはおそらく、空中では働いている力が、机の上にボールを置いたことで重力と釣り合う力が働いていることになります。
高校時代にこれは、机からの\textbf{垂直抗力}と呼んでいたものです。
このように「質点の運動を束縛する力」が働いていることもあります。
これをそのまま、\textbf{束縛力}と呼びます。

さらにもう一つ、\textbf{質点の速さそのものによって受ける力}があります。
身近な例であれば、雨の落下時にかかる空気摩擦です。
空気摩擦は雨の落下が速くなればなるほど強くかかってしまうような気がします。
雨の例のほかにも、電磁気学におけるローレンツ力というのがあり、磁場中を荷電粒子が動くと、思わぬ方向に力が働くのです。

このように色々な力があるんですが、大きく分けて2種類あると考えましょう。
\begin{itemize}
  \item 外力$F$とは、$I\times\mathbb{R}^3\times\mathbb{R}^3$の開集合$U$から$\mathbb{R}^3$への連続写像
  \item 束縛力$F_c$は、観測事実から逆算ないし予測して追加されるベクトル量
\end{itemize}
質点$(I,\gamma)$に外力$F$がかかっている様子は、
\[
  F(t,\gamma(t),\dot\gamma(t))
\]
で得られます。
束縛力に関しては、少々面倒なので、暫くの間は束縛力はないものと考えて差し支えありません。

ちなみに、今まで「力によって加速度が変わる」とも取れる言い方もしていましたが、実際のところ本当にそうなのかはわかりません。
この節の冒頭にも述べたように、「加速度が変わると力が働いたように『見える』」というだけで、力を定義したに過ぎないです。
人間がそう感じるというだけで、自然法則が本当にそう動いているかは分かりません。
ここまで言っておいてなんですが、実際のところ力とは一体何なのでしょうか?



\section{運動の法則}

力の定義が済んだところで、力と質点の運動の関係についての法則を定義していきましょう。

\subsection{ガリレイの相対性原理と慣性系}

さて僕たちはこれから、質点と力の間の関係を定式化して、質点の未来予測を可能とする理論を建てようとしています。
その前に一点、力の定義の穴を埋めておきましょう。

力の定義は、「加速度が変化するときに力が働いているように感じる」という人間本位の考えに基づいて力を定義しましたが、ある見方をすると「ある人から見ると加速度が変化するように見え、別の人から見ると加速度が変化していないように見える」ということが起こりえることを考慮していませんでした。
例えばメリーゴーランドに乗っている人から見て静止している質点は、メリーゴーランドの外から見ている人には回転運動という加速度が変化する運動をしています。
これでは力の定義というのが、見ている人、すなわち\textbf{観測者}によって異なってしまいます。

観測者がどのような運動をしているかによって、基本的な運動法則の記法が変わってしまっては、何かをすり抜けてしまったように感じませんでしょうか?
じゃあ観測者の運動に依存して、それぞれ運動法則を書けばいいでしょうか?
いや、それも…カッコつけて言えば、オッカムの剃刀的に違和感があります。

そこで、「メリーゴーランドに乗ってる」というような奇妙な状況は一旦排除して、基礎的な運動法則を打ち立てましょう。
力学はもともと、天体の運行を調べるための、自然科学の数学的諸原理を記述するためにニュートンが作り出したものなのです。
ということで、まず宇宙に対して静止している系というものの存在を仮定します。
\begin{definition}
  射影
  \begin{align*}
    t:\mathbb{R}\times\mathbb{R}^3;(t,(x,y,z))\mapsto t\\
    x:\mathbb{R}\times\mathbb{R}^3;(t,(x,y,z))\mapsto x\\
    y:\mathbb{R}\times\mathbb{R}^3;(t,(x,y,z))\mapsto y\\
    z:\mathbb{R}\times\mathbb{R}^3;(t,(x,y,z))\mapsto z
  \end{align*}
  の組$(t,x,y,z)$を\textbf{標準慣性系}とよぶ。
\end{definition}

次にガリレイ変換を定義します。
\begin{definition}[ガリレイ変換と慣性系]
  $(t,x,y,z)$を標準慣性系、$t'$を実数、$(t_0,x_0,y_0,z_0)\in\mathbb{R}^4, (v_x,v_y,v_z)\in\mathbb{R}^3$をベクトル、また
  \[
    R:=
    \begin{pmatrix}
      R_{11} & R_{12} & R_{13} \\
      R_{21} & R_{22} & R_{23} \\
      R_{31} & R_{32} & R_{33}
    \end{pmatrix}
  \]
  は回転行列\footnote{
    $R$をこの行列としたとき、縦に並んでいる3本のベクトルが互いに直交していて長さが1、かつ、行列式が1なものを(3次の)回転行列といいます。
    具体的には
    \[
      \begin{pmatrix}
        \cos\theta_1 & -\sin\theta_1 & 0 \\
        \sin\theta_1 & \cos\theta_1 & 0 \\
        0 & 0 & 1
      \end{pmatrix},\quad
      \begin{pmatrix}
        \cos\theta_1 & 0 & -\sin\theta_1 \\
        0 & 1 & 0 \\
        \sin\theta_1 & 0 & \cos\theta_1
      \end{pmatrix},\quad
      \begin{pmatrix}
        1 & 0 & 0 \\
        0 & \cos\theta_1 & -\sin\theta_1 \\
        0 & \sin\theta_1 & \cos\theta_1
      \end{pmatrix},\quad
    \]
    によって生成されます。
  }であるとする。
  このとき、写像
  \[
    \begin{pmatrix}
      t'\\x'\\y'\\z'
    \end{pmatrix}
    =
    \begin{pmatrix}
      1 & 0 & 0 & 0 \\
      v_1 & R_{11} & R_{12} & R_{13} \\
      v_2 & R_{21} & R_{22} & R_{23} \\
      v_3 & R_{31} & R_{32} & R_{33}
    \end{pmatrix}
    \begin{pmatrix}
      t\\x\\y\\z
    \end{pmatrix}
    +
    \begin{pmatrix}
      t_0\\x_0\\y_0\\z_0
    \end{pmatrix}
  \]
  を\textbf{ガリレイ変換}といい、ガリレイ変換後の座標系
  \begin{align*}
    t'&=t+t_0\\
    x'&=v_1t+R_{11}x+R_{12}y+R_{13}z+x_0\\
    y'&=v_2t+R_{21}x+R_{22}y+R_{23}z+y_0\\
    z'&=v_3t+R_{31}x+R_{32}y+R_{33}z+z_0
  \end{align*}
  を\textbf{慣性系}という。
\end{definition}

\begin{example}
  簡単のため回転行列を$R=1$(単位行列)、$v_2=v_3=0$とする。
  このとき、ガリレイ変換は
  \begin{align*}
    t'&=t+t_0\\
    x'&=v_1t+x\\
    y'&=y\\
    y'&=z
  \end{align*}
  となります。これは、$x$軸上をトコトコ走る列車に乗った実験室をイメージしたものになっています。
\end{example}

「力学の法則は慣性系の違いにおいて不変なように記述すべし」と高らかに宣言するのが、ガリレイの相対性原理です。
\begin{definition}[ガリレイの相対性原理]
  力学の諸法則は、慣性系の違いにおいて理論の形を変えないべきである。
\end{definition}
よく磨いた剃刀ですな。
まあ、あとでアインシュタインがひっくり返すんですけどね。

\subsection{慣性の法則}

慣性系やガリレイの相対性原理のおかげで、慣性の法則を慣性系の範囲で語ることができます。
まず、自由粒子というものを定義します。
\begin{definition}
  ある慣性系において、合力0の力がかかっている質点を\textbf{自由粒子}と呼ぶ。
\end{definition}
\begin{definition}[慣性の法則]
  自由粒子は(静止を含めた)等速直線運動をする。
  数式で書けば、ある慣性系$(t,x,y,z)$において、自由粒子$(I,\gamma)$は
  \[
    \ddot\gamma(t)=0
  \]
  を満たす。
\end{definition}

慣性の法則がガリレイの相対性原理を満たしていることを確認しましょう。
質点の運動に関してガリレイ変換をすると
\[
  \begin{pmatrix}
    t'\\
    \gamma'_x(t')\\
    \gamma'_y(t')\\
    \gamma'_z(t')
  \end{pmatrix}
  =
  \begin{pmatrix}
    t+t_0\\
    v_1t+R_{11}\gamma_x(t)+R_{12}\gamma_y(t)+R_{13}\gamma_z(t)+x_0\\
    v_2t+R_{21}\gamma_x(t)+R_{22}\gamma_y(t)+R_{23}\gamma_z(t)+y_0\\
    v_3t+R_{31}\gamma_x(t)+R_{32}\gamma_y(t)+R_{33}\gamma_z(t)+z_0
  \end{pmatrix}
\]
というふうにガラガラ変わってしまいます。
ところがこれの1回微分は
\begin{align*}
  \frac{d\gamma'(t')}{dt'}&=\frac{dt}{dt'}\frac{d\gamma'(t')}{dt}=\frac{d(t'-t_0)}{dt'}\frac{d\gamma'(t')}{dt}=\frac{d\gamma'(t')}{dt}\\
  &=\frac{d}{dt}
  \begin{pmatrix}
    v_1t+R_{11}\gamma_x(t)+R_{12}\gamma_y(t)+R_{13}\gamma_z(t)+x_0\\
    v_2t+R_{21}\gamma_x(t)+R_{22}\gamma_y(t)+R_{23}\gamma_z(t)+y_0\\
    v_3t+R_{31}\gamma_x(t)+R_{32}\gamma_y(t)+R_{33}\gamma_z(t)+z_0
  \end{pmatrix}\\
  &=
  \begin{pmatrix}
    v_1+R_{11}\dot\gamma_x(t)+R_{12}\dot\gamma_y(t)+R_{13}\dot\gamma_z(t)\\
    v_2+R_{21}\dot\gamma_x(t)+R_{22}\dot\gamma_y(t)+R_{23}\dot\gamma_z(t)\\
    v_3+R_{31}\dot\gamma_x(t)+R_{32}\dot\gamma_y(t)+R_{33}\dot\gamma_z(t)
  \end{pmatrix}
\end{align*}
となり、もう一度微分すると
\begin{align*}
  \frac{d^2\gamma'(t')}{dt'^2}&=\frac{d}{dt}
  \begin{pmatrix}
    v_1+R_{11}\dot\gamma_x(t)+R_{12}\dot\gamma_y(t)+R_{13}\dot\gamma_z(t)\\
    v_2+R_{21}\dot\gamma_x(t)+R_{22}\dot\gamma_y(t)+R_{23}\dot\gamma_z(t)\\
    v_3+R_{31}\dot\gamma_x(t)+R_{32}\dot\gamma_y(t)+R_{33}\dot\gamma_z(t)
  \end{pmatrix}
  =
  \begin{pmatrix}
    R_{11}\ddot\gamma_x(t)+R_{12}\ddot\gamma_y(t)+R_{13}\ddot\gamma_z(t)\\
    R_{21}\ddot\gamma_x(t)+R_{22}\ddot\gamma_y(t)+R_{23}\ddot\gamma_z(t)\\
    R_{31}\ddot\gamma_x(t)+R_{32}\ddot\gamma_y(t)+R_{33}\ddot\gamma_z(t)
  \end{pmatrix}\\
  &=R\frac{d^2\gamma(t)}{dt^2}
\end{align*}
を満たすので、「質点の二階微分が消える」という性質はガリレイ変換で保たれており\footnote{
  回転行列はもちろん正則行列だから。
}、慣性の法則はガリレイの相対性原理を満たしています。
これは、慣性の法則がちゃんと力学の法則なってますよと保証してくれています。

\subsection{運動方程式}

僕たちの直感によれば、質点の加速度が変化と、力の間に関係式があるはずだというモデルを作る妥当性があるのでした。
ところが、例えば重さの違う二つの大玉を押すことを考えてみましょう。
この大玉を時速10kmの速さまで加速させるのに必要な「大変さ」は、それぞれ異なります。
「重さ」の定義はまだ行っていないですが、どのみち実体験として、「加速させる大変さが異なる物体がある」というのは実体験としてあります。
この「加速する大変さ」は、このような考察から質点に依存しているはずです。
そこで質点に対して一つパラメータを追加しましょう。

\begin{definition}[慣性質量]
  0以上の実数$m$と質点$\gamma:I\to\mathbb{R}^3$の組$(\gamma,m)$を、\textbf{質量付き質点}といい、$m$をその\textbf{(慣性)質量}という。
  誤解のないときは、質量付き質点を単に質点と呼び、$\gamma:=(I,\gamma,m)$であらわす。
\end{definition}

これで力学の最も基本的な法則を定義することができます。
\begin{definition}[運動方程式]
  $I$を区間、$U\subset I\times\mathbb{R}^3\times\mathbb{R}^3$を開集合とする。
  質量付き質点$(I,\gamma,m)$と外力$F:U\to\mathbb{R}^3$に対して、
  \[
    F(t,\gamma(t),\dot\gamma(t))=m\ddot\gamma(t)
  \]
  が成り立つ。
\end{definition}

この等式はあくまで「加速度が変わったら力がかかった!」ないし「力によって加速度が変わったとみなせる」という身勝手な直感をもとに建てた式です。
ゆえに、実は「質点の加速度×質量が質点にかかっている力」なのか、「力によって質点の速度が変わる」のかは微妙な問題です。
鶏と卵の問題というものです。
ニュートン力学が当たり前に受け入れられている現代では、質点の加速度と力の間に運動方程式のような関係があってもおかしく感じないかもしれませんが、それ以前の科学者・哲学者はそれはそれは色々なことを考えたことでしょう。
ニュートン以降は「この両者の関係がこのように定まっていると仮定する」としている感覚に近いかもしれません。
少なくとも『本当にそうなのか?そもそも「力」って何?』という哲学っぽい問題にまで踏み込んでないです。
ともあれこの前提によって、不思議なことに色々な現実の問題を解決することができるのです。

ところで、運動方程式の定義における力は外力のみ考えています。
実際には垂直抗力などの$F(t,\gamma(t),\dot\gamma(t))$ではとらえきれない、質点の運動を実験によって観測してモデル化しないと定式できない力もあるのでした。
現実問題を考えるに当たっては、例えば空気抵抗のように、左辺にそのような項を加えて方程式を解いていきます。
しかし「力学の基本原理」という意味では、そのような力を加えるべきではないでしょう。

力学の基本原理といえば、ガリレイ変換で不変であるべきなのでした。
そこのところどうなのか、見ていきましょう。

\subsection{力のガリレイ変換}

加速度$\ddot\gamma(t)$はガリレイ変換
\[
  \begin{pmatrix}
    t'\\x'\\y'\\z'
  \end{pmatrix}
  =
  \begin{pmatrix}
    1 & 0 & 0 & 0 \\
    v_1 & R_{11} & R_{12} & R_{13} \\
    v_2 & R_{21} & R_{22} & R_{23} \\
    v_3 & R_{31} & R_{32} & R_{33}
  \end{pmatrix}
  \begin{pmatrix}
    t\\x\\y\\z
  \end{pmatrix}
  +
  \begin{pmatrix}
    t_0\\x_0\\y_0\\z_0
  \end{pmatrix}
\]
に対して、
\[
  \frac{d^2\gamma'(t')}{dt'}=R\frac{d^2\gamma(t)}{dt^2}
\]
と変換することを見ています。
「ガリレイ変換で不変な法則のみが根本的な力学の原理である」とする立場を厳守するならば、運動方程式
\[
  F(t,\gamma(t),\dot\gamma(t))=m\ddot\gamma(t)
\]
の左辺もガリレイ変換後、ガリレイ変換の回転行列部分$R$が掛かる形で変換されねばなりません。
\[
  F(t',(x',y',z'),(\dot{x'},\dot{y'},\dot{y'}))=RF(t,(x,y,z),(\dot{x},\dot{y},\dot{z}))
\]
実際のところ、運動方程式においては上記を仮定に加えることにします。
\begin{definition}
  $F$を外力とする。
  ガリレイ変換
  \[
    \begin{pmatrix}
      t'\\x'\\y'\\z'
    \end{pmatrix}
    =
    \begin{pmatrix}
      1 & 0 & 0 & 0 \\
      v_1 & R_{11} & R_{12} & R_{13} \\
      v_2 & R_{21} & R_{22} & R_{23} \\
      v_3 & R_{31} & R_{32} & R_{33}
    \end{pmatrix}
    \begin{pmatrix}
      t\\x\\y\\z
    \end{pmatrix}
    +
    \begin{pmatrix}
      t_0\\x_0\\y_0\\z_0
    \end{pmatrix}
  \]
  に対して、
  \[
    F(t',(x',y',z'),(\dot{x'},\dot{y'},\dot{y'}))=RF(t,(x,y,z),(\dot{x},\dot{y},\dot{z}))
  \]
  と変換する外力を\textbf{ガリレイ不変な外力}または\textbf{真の力}\footnote{
    コリオリ力などの「見かけの力」に対応しての命名。使うかどうかは未定。
  }と呼ぶ。
\end{definition}

これによって運動方程式を修正しましょう。
\begin{definition}[運動方程式]
  $I$を区間、$U\subset I\times\mathbb{R}^3\times\mathbb{R}^3$を開集合とする。
  質量付き質点$(I,\gamma,m)$とガリレイ不変な外力$F:U\to\mathbb{R}^3$に対して、
  \[
    F(t,\gamma(t),\dot\gamma(t))=m\ddot\gamma(t)
  \]
  が成り立つ。
\end{definition}

これはあくまで基本原理であって、宇宙の法則はこれに従っているだろうというニュートンのモデリングにすぎません(多分)。
なのでもちろん、例えば空気抵抗ありの落下運動のように、ガリレイ不変とは限らない力を考慮して、運動方程式を立てて解くという場面もよく見かけます。
