\chapter{運動の法則}

質点に対して、いくつかの仮定(公理)を追加します。
力学はここから始まるのです。
まずは運動の状況を見ていきましょう。



\section{簡単な運動}

最も簡単な運動は「静止」している状態でしょう。
\begin{definition}
  質点の運動$\gamma:I\to\mathbb{R}^3$が定値写像になっているとき、質点$\gamma$は\textbf{静止}しているという。
  静止している質点の運動は、静止であると言い表す。
\end{definition}

静止の次に簡単な運動は、速さが変化しない運動、つまり「等速直線運動」でしょう。
\begin{definition}
  質点の運動$\gamma:I\to\mathbb{R}^3$に対し、$\dot\gamma:I\to\mathbb{R}^3$が定値写像になっているとき、質点$\gamma$は\textbf{等速直線運動}しているという。
  等速直線運動している質点の運動は、等速直線運動であると言い表す。
\end{definition}

質点が静止しているならば、等速直線運動しているということが言えます。
\begin{theorem}
  質点$\gamma:I\to\mathbb{R}^3$の運動が静止であるならば、その運動は等速直線運動である。
\end{theorem}
\begin{proof}
  $\gamma$は静止しているので、任意の2時刻$t_0,t\in I$において$\gamma(t)=\gamma(t_0)$。
  ゆえに
  \[
    \dot\gamma(t_0)=\lim_{t\to t_0}\frac{\gamma(t)-\gamma(t_0)}{t-t_0}=\lim_{t\to t_0}0=0
  \]
  よって$\gamma$は等速直線運動である。
\end{proof}

また、等速直線運動する質点の加速度$\ddot\gamma$は0への定値写像であることが同値であることも言えます。
ここには\textbf{平均値の定理}を用いるのですが、それはのち(§\ref{sec:mean-value-theorem})に証明しましょう。
いったん平均値の定理は認めて、以下を証明しましょう。
\begin{theorem}
  質点$\gamma$の運動が等速直線運動であることと、任意の$t_0\in I$に対して$\ddot\gamma(t_0)=0$は同値である。
\end{theorem}
\begin{proof}
  質点$\gamma$の運動が等速直線運動であるとすると、任意の$t,t_0\in I$に対して、$\dot\gamma(t)=\dot\gamma(t_0)$。
  ゆえに
  \[
    \ddot\gamma(t_0)\lim_{t\to t_0}\frac{\dot\gamma(t)-\dot\gamma(t_0)}{t-t_0}=0
  \]
  ゆえに$\ddot\gamma(t_0)=0$ (${}^\forall t_0\in I$)。

  逆に$\ddot\gamma(t_0)=0$ (${}^\forall t_0\in I$)とする。
  $\gamma$を成分ごとに分けて、$\gamma(t):=(\gamma_x(t),\gamma_y(t),\gamma_z(t))$で考える。
  このとき任意の$t_1,t_2\in I$、$t_1<t_2$に対して、平均値の定理から、ある$c\in(t_1,t_2)$が存在して、
  \[
    \frac{\dot\gamma_i(t_1)-\dot\gamma_i(t_2)}{t_1-t_2}=\ddot\gamma_i(c)=0,\quad(i=x,y,z)
  \]
  をみたす。すなわち$\dot\gamma_i(t_1)=\dot\gamma_i(t_2)$なので、$\dot\gamma_i(t)$は定数写像となる。
\end{proof}



\section{運動の法則}

「力」という概念の定義は非常に難しいものです。
例えば僕たちは地球の重力が働いていると知っていますが、重力があるからといって地球の中心に向かって落ち続けるということはないです。
それは重力と地面からの垂直抗力が釣り合っているから、と説明できます。
しかしそのように説明をもとに色々な現象を実証していくのは、実際には容易ではありません。

アリストテレスは「槍を投げると、飛んでいる間は運動方向に力が働き続ける。その力が徐々に失われて行って、しまいには地面に落ちる」というような考え方をしていたようです。

実際のところ「力」って何なのさ?という問題は哲学の分野で語ってもらうことにして、物理学ではとにかく\textbf{定量化して計算できること}を重視します。
言い換えれば、そのように現実世界を「モデリング」して、定量的な計算や未来予測を可能にしようぜ、というのがニュートンの思想の一端ではないでしょうか。

\subsection{慣性の法則}

これまで静止する質点や、等速直線運動する質点を見てきました。
果たしてこれらに力はかかっていると言えましょうか?(いや、ない。)
ということを踏まえた、「力がかかっていない」というニュートン流の定義がこちらになります。
\begin{definition}[慣性の法則]
  質点が等速直線運動するとき、またそのときに限って\textbf{質点には力が働いていない}と言い表す。
\end{definition}

これは力が働いていないから等速直線運動するという因果関係…ではなく、「力が働いていないこと」の定義です。
「等速直線運動しているからこそ『力が働いていない』と\textbf{定義した}」んです。
気を付けてくださいね。

\subsubsection{ガリレイの相対性原理}

物理学では「観測の対象となる物理現象」の他に、実は「それを観測する主体(観測者)」の二者が存在しています。
しかも観測者というのは複数存在しえます。

例えば雨の落下という物理現象を考えましょう。
地表に対して静止している観測者にとっては、(風がなければ)雨は地面に対して垂直に落下しているように見えます。
ところが等速直線運動している列車に乗っている観測者からは、斜めに落ちていくように見えます。
当の物理現象である雨の落下が、観測者によって物理法則が変わってしまうことを意味するかというと、答えはNOです。
たしかに見かけの運動は変わったかに見えますが、それは地面上でも列車内でも同じ法則によって記述されうる現象ではあります。
そんなぐらいの原理を\textbf{ガリレイの相対性原理}といいます。

より詳細にガリレイの相対性原理に迫るため、まずは座標変換について次のように定式しましょう。
\begin{definition}[ガリレイ変換]
  $A:\mathbb{R}\times\mathbb{R}^3\to\mathbb{R}^3$が、回転行列$R$\footnote{
    普通の2次元の回転行列を3次元に拡大して、よしなに0で埋めたやつです。$SO(3)$の元と言ったほうが早い人もいましょう。
  }、ベクトル$(v_x,v_y,v_z),(x_0,y_0),z_0$に対して
  \[
    A(t,x,y,z):=(x,y,z)R+(v_x,v_y,v_z)t+(x_0,y_0,z_0)
  \]
  と書けるとき、$A$を\textbf{ガリレイ変換}という。
  通常のユークリッド空間から、ガリレイ変換で移りあう座標系は\textbf{慣性系}と呼ばれ、これも記号$A$で表される。
\end{definition}

慣性系$A$と質点の運動$(I,\gamma)$に対して、
\[
  \gamma_A:I\to\mathbb{R}^3;t\mapsto A(t,\gamma(t))
\]
とおくことにします。
この準備で、等速直線運動という質点の性質が、以下の意味でガリレイ変換で保たれることがわかります。
\begin{theorem}
  任意の質点$(I,\gamma)$が等速直線運動するとき、任意の慣性系$A$に対して、$(I,\gamma_A)$も等速直線運動する。
\end{theorem}

ガリレイの相対性原理は、「ガリレイ変換によって変わる観測者ごとに同じ観測を行っても、結局どれも同じ結果になるだろう」という話です。
そのうち「力が働いていない版」として、慣性の法則を定式し直すと以下のようになりましょう。
\begin{definition}[慣性の法則(慣性系)]
  質点$(I,\gamma)$に力が働いていないとは、ある慣性系$A$が存在して、$(I,\gamma_A)$が等速直線運動するときをいう。
\end{definition}



\subsection{運動方程式}

ここをしっかり定義すると、力の正体にあまり深入りすることなく、この定義の対偶をとって次のように考えることができます。
つまり、力が働いているということは、等速直線運動しないということです。
等速直線運動しない具合を力と呼ぶべきであって、そうすることで力を定量的に測ることができます。
つまり、速さの変化の割合…加速度が、その質点にかかっている力を表しているはずです。

ところが、例えば重さの違う二つの大玉を押すことを考えてみましょう。
この大玉を時速10kmの速さまで加速させるのに必要な「大変さ」は、それぞれ異なります。
「重さ」の定義はまだ行っていないですが、どのみち実体験として、「加速させる大変さが異なる物体がある」というのは実体験としてあります。
この「加速する大変さ」は、このような考察から質点に依存しているはずです。
そこで質点に対して一つパラメータを追加しましょう。

\begin{definition}[慣性質量]
  0以上の実数$m$と質点$\gamma:I\to\mathbb{R}^3$の組$(\gamma,m)$を、\textbf{質量付き質点}といい、$m$をその\textbf{(慣性)質量}という。
  誤解のないときは、質量付き質点を単に質点と呼び、$\gamma:=(I,\gamma,m)$であらわす。
\end{definition}

これで力学の最も基本的な法則を定義することができます。
\begin{definition}[運動方程式]
  質量付き質点$(I,\gamma,m)$に対して、
  \[
    F:=m\ddot\gamma:I\to\mathbb{R}^3
  \]
  を、質点$(\gamma,m)$にかかる力と呼ぶ。
\end{definition}

このように、いったん力は質点に対して上記のように定義しました。
普通は上記とは逆に、質点にかかっている力を先に考えて、上記の運動方程式にあてはめることで質点の運動を予測します。
そのように考える場合、力と運動方程式は次のように再定義することになります。
\begin{definition}[運動方程式]
  $I\subset\mathbb{R}$を区間、$U\subset\mathbb{R}^3$を部分位相空間とする。連続写像
  \[
    F:I\times U\to\mathbb{R}^3
  \]
  を\textbf{力(力場)}と呼ぶ。
  また、質量付き質点$(I,\gamma,m)$に対して、
  \[
    m\ddot\gamma=F(t,\gamma(t))
  \]
  を質点$(\gamma,m)$の\textbf{運動方程式}と呼ぶ。
  上記の運動方程式を満たす質点は、力場$F$に\textbf{束縛される}という。
\end{definition}

このように「質点の加速度×質量が質点にかかっている力」なのか、「力によって質点の速度が変わる」のかは実は微妙な問題です。
鶏と卵の問題というものです。
ニュートン力学が当たり前に受け入れられている現代では、質点の加速度と力の間に運動方程式のような関係があってもおかしく感じないかもしれませんが、それ以前の科学者・哲学者はそれはそれは色々なことを考えたことでしょう。
ニュートン以降は「この両者の関係がこのように定まっていると仮定する」としている感覚に近いかもしれません。
少なくとも『本当にそうなのか?そもそも「力」って何?』という哲学っぽい問題にまで踏み込んでないです。
ともあれこの前提によって、不思議なことに色々な現実の問題を解決することができるのです。
