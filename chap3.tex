\chapter{微分の基礎}

せっかく微分がでてきたので、数学的にしっかり基礎づけしていきましょう。



\section{定義のおさらい}

\begin{definition}
  $I$を区間、$f:I\to\mathbb{R}$を(連続とは限らない)関数、$x\in I$とする。
  このとき、極限
  \[
    f'(x):=\lim_{y\to x}\frac{f(x)-f(y)}{x-y}
  \]
  が存在するとき、$f'(x)$を$f$の$x$における\textbf{微分係数}、あるいは単に\textbf{微分}という。
\end{definition}

\begin{definition}
  $I$を区間、$f:I\to\mathbb{R}$を(連続とは限らない)関数とする。
  \textbf{$I$上微分可能である}とは、任意の$x\in I$に対して、微分係数$f'(x)$が存在する時を言う。
  このとき、関数
  \[
    f':I\to\mathbb{R}
  \]
  を、$f$の\textbf{導関数}、あるいは単に\textbf{微分}という。
\end{definition}

$y\to x$を、$y=x+h$の変数変換によって
\[
  f'(x)=\lim_{h\to0}\frac{f(x+h)-f(x)}{h}
\]
と書き換えることも多いし、なんなら高校の教科書での定義はこっちですね。



\section{例題集}

\subsection{べき関数}

まずは簡単なところから行きましょう。
$n>0$を整数として、
\[
  f(x)=x^n
\]
を微分してゆきましょう。
このとき
\[
  \frac{(x+h)^n-x^n}{h}=nx^{n-1}+\frac{n(n-1)}{2}hx^{n-2}+\cdots+h^{n-1}\to nx^{n-1}\quad\text{as }h\to 0
\]
ゆえに
\[
  (x^n)'=nx^{n-1}
\]
です。



\section{微分の性質}

\subsection{線形性}

\subsection{積の微分}

\subsection{商の微分}

\subsection{合成関数の微分}

\section{平均値の定理}

\section{テイラー展開}
