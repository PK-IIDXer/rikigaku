\chapter{運動方程式を解く}

ニュートンの第二法則・運動方程式は現状、「力と質点の運動の関係ををこう定義してみた」という以上のものではないです。
本章では、いくつか現実の問題に対して運動方程式を立てて、「質点の運動を予測する」というところまでやっていきましょう。



\section{自由粒子}

\begin{definition}[自由粒子]
  質点$(I,\gamma,m)$が力場$F(t,x,y,z)=0$に束縛されるとき、その質点を\textbf{自由粒子}という。
\end{definition}

自由粒子の運動は慣性の法則によって等速直線運動をする…と思われるかもしれませんが、これは妥当ではありません。
慣性の法則は「等速直線運動する質点には力が働いていない」とする仮定なので、自由粒子が等速直線運動するかどうかはまだ分からないのです。
ということで、自由粒子の運動を調べていきましょう。

自由粒子の運動方程式は$m\gamma''(t)=0$ですが、$m$は定数なので
\[
  \gamma''(t)=0
\]
を考えればよいです。
これは前章で見た通り、$\gamma'$が定値写像になることと同値な等式です。
すなわち等速直線運動です。

さて、
\[
  \gamma'(t)=v_0\in\mathbb{R}^3
\]
とおいて、$\gamma$を求めていきましょう。
再び平均値の定理を使って、任意の$a<b$に対して、ある$c\in(a,b)$が存在して、
\[
  \frac{\gamma(b)-\gamma(a)}{b-a}=\gamma'(c)=v_0
\]
が成り立ちます。
すなわち
\[
  \gamma(b)=v_0(b-a)+\gamma(a)
\]
です。
ここから少し直感的に議論しますが、$b$がめちゃくちゃ$a$に近い実数$b=a+\Delta t$であるとしましょう。
\[
  \gamma(a+\Delta t)=v_0\Delta t+\gamma(a)
\]
この式から読み取れるのは、「$a$から少しだけ値をずらした値が、$v_0\Delta t+\gamma(a)$で得られる」です。
もし$a$を$a+\Delta t$に、$a+\Delta t$を$a+2\Delta t$にしても同様に
\begin{align*}
  \gamma(a+2\Delta t)&=v_0\Delta t+\gamma(a+\Delta t)\\
  &=2v_0\Delta t+\gamma(a)
\end{align*}
などが成り立ちます。
$a$からの微小変化$\Delta t$の大きさに比例して、$v_0$の係数が増えていくので、多分こうでしょう。
\[
  \gamma(t)=v_0t+\gamma(a)
\]

…もう少しまともにみましょう。
積分の登場です。



\section{積分と微分積分学の基本定理}

閉区間$I=[a,b]$上の関数$f:I\to\mathbb{R}$に対して、始点$a$から微小な区間$\Delta x$を積み上げて$t\in I$まで足し上げることを考えます。
より一般的な場合を考えて、$a$から$t$までの閉区間の分割の取り方を次のようにしましょう。
\begin{definition}
  閉区間$[a,t]$に対して、数列$x_0,x_1,\dots,x_{n+1}$が
  \[
    a=x_0<x_1<x_2<\cdots<x_{n+t}=t
  \]
  を満たす時、$\Delta:=\{x_0,\dots,x_{n+1}\}$を閉区間$[a,t]$の\textbf{分割}という。
\end{definition}
既に動機は示しているので、次のようなことを考えたいですね。
つまり、$\Delta:=\{x_0,\dots,x_{n+1}\}$を閉区間$[a,t]$の分割としたとき、
\[
  F_\Delta:=\sum_{i=0}^n f(c_i)(x_{i+1}-x_i)
\]
ただし$c_i\in[x_i,x_{i+1}]$としています。
これを$x_{i+t}-x_i$が0に向かいつつ、区間をギャンギャンに細分する極限を考えたいです。
その様子を一挙に表すために、分割の幅を考えます。
\begin{definition}
  $\Delta:=\{x_0,\dots,x_{n+1}\}$を閉区間$[a,t]$の分割としたとき、
  \[
    |\Delta|:=\max\{x_{i+1}-x_i\mid i=0,\dots,n\}
  \]
  を、分割$\Delta$の\textbf{幅}という。
\end{definition}
分割$\Delta$の幅を0に向かわせる極限$|\Delta|\to0$を考えることで、区間の幅を狭めつつ細分も細かくするという状況を考えることができるわけです。

これで積分を定義する準備が整いました。
\begin{definition}
  $I:=[a,b]$を閉区間、$t\in I$、$f:I\to\mathbb{R}$を(連続とは限らない)関数とする。
  $\Delta:=\{x_0,\dots,x_{n+1}\}$を閉区間$[a,t]$の任意の分割とする。
  このとき、極限
  \[
    \lim_{|\Delta|\to0}\sum_{i=0}^n f(c_i)(x_{i+1}-x_i)
  \]
  が、$c_0,\dots,c_n$の選び方によらず収束するとき、関数$f$は$[a,t]$で\textbf{積分可能}であるといい、極限
  \[
    \int_a^tf(x)dx:=\lim_{|\Delta|\to0}\sum_{i=0}^n f(c_i)(x_{i+1}-x_i)
  \]
  と書き、$f$の$a$から$t$への\textbf{(リーマン)積分}とよぶ。
\end{definition}

定義はできたものの、これがいつ収束するのかを調べておく必要があります。
実は比較的広い状況で積分可能であることが以下のように示せます。
\begin{theorem}
  関数$f$が閉区間$[a,t]$で連続ならば、その閉区間で$f$は積分可能である。
\end{theorem}
\begin{proof}
  TODO: 一様連続だから。
\end{proof}

もともと「微分して$\gamma'(t)=v_0$となる$\gamma$を求めたい」ということでここまで進めてきました。
実際、しっかりこれを証明することができます。
\begin{theorem}
  微分積分学の基本定理
\end{theorem}
\begin{proof}
  TODO
\end{proof}
\begin{corollary}
  $F:[a,b]\to\mathbb{R}$を$C^1$級関数とし、$f:=F'$とする。このとき
  \[
    \int_a^b f(x)dx=F(b)-F(a)
  \]
\end{corollary}

微分積分学の基本定理によって、ある関数$f$の積分を求めるためには、微分して$f$になる関数$F$を求めれば良いことがわかりました。
つまり、積分は微分の逆演算であるということです。
ここに、積分の結果が関数なのか実数なのか分からなくなることが出てきたので、用語を定義しておきます。
\begin{definition}
  $F:[a,b]\to\mathbb{R}$を$C^1$級関数とし、$f:=F'$とする。
  このとき、$F$を$f$の\textbf{原始関数}または\textbf{不定積分}と呼び、
  \[
    F(x):=\int f(x)dx
  \]
  と書く。また、
  \[
    \int_a^b f(x)dx
  \]
  を、$f$の\textbf{($a$から$b$にかける)定積分}と呼ぶ。
\end{definition}



\section{有名な微分と積分}



\section{一様な力場の運動}

\begin{definition}
  定値写像であるような力場
  \[
    F:I\times\mathbb{R}^3\to\{f\}\subset\mathbb{R}^3
  \]
  を、\textbf{一様な力場}という。
\end{definition}

例として、$a>0$に対して、一様な力場$F(t,x,y,z)=(0,0,-a)$に束縛される質点の運動を調べましょう。
\[
  \gamma(t):=(\gamma_x(t),\gamma_y(t),\gamma_z(t))
\]
とおいて、成分ごとに調べていきましょう。
